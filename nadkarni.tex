\documentclass[a4paper]{article}

\usepackage[T1]{fontenc}
\usepackage[utf8]{inputenc}
\usepackage{lmodern}

\usepackage[UKenglish]{babel}
\usepackage[UKenglish]{isodate}
\usepackage[colorlinks=true,urlcolor=blue]{hyperref}
\usepackage{incgraph}
\usepackage[all]{genealogytree}

\hypersetup{
	pdftitle={Bendkar Nadkarni Family Tree},
	pdfauthor={Viren Nadkarni}
}

\begin{document}
	\begin{inctext}
	\begin{tikzpicture}
		\genealogytreeinput[
			template=database traditional,
		]{nadkarni.graph}
		
		\path
		([xshift=-1cm,yshift=-1.16cm]current bounding box.south west)
		rectangle ([xshift=1.16cm,yshift=1cm]current bounding box.north east);
		
		\path[draw=black,line width=1pt,rounded corners=6pt]
		([xshift=3mm,yshift=3mm]current bounding box.south west)
		rectangle ([xshift=-3mm,yshift=-3mm]current bounding box.north east);
		
		\node[below right] at ([xshift=1cm,yshift=-1cm]current bounding box.north west) {
			\begin{tcolorbox}[
				enhanced,
				width=12cm,
				boxrule=0.4pt,
				arc=2pt,
				colframe=black,
				segmentation style={solid,shorten >=3mm,shorten <=3mm},
			]
				\begin{center}
					\bfseries\Large\color{black} Bendkar Nadkarni
				\end{center}

				\href{https://github.com/viren-nadkarni/nadkarni-family-tree}{https://github.com/viren-nadkarni/nadkarni-family-tree} \\
				Version 2022-12-10
			\end{tcolorbox}
		};		
	\end{tikzpicture}
\end{inctext}

\end{document}
